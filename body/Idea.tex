\chapter{任务需求分析}
任务需求分析是指对任务需求进行系统的分析和整理,以明确任务的目标、范围和具体要求。通过需求分析,可以有效地确定任务的可行性和可实施性,并为后续的开发、测试和评估提供重要参考。

\section{第一周:实现简单的 Echo Web Server}
实现简单的 Echo Web Server是一种基于HTTP协议的Web应用程序,它可以接收common请求,并将请求数据原样返回给客户端。通过实现Echo Web Server,可以深入理解Web Server的工作原理和相关技术,同时也能够提高编程技能和动手能力。该应用程序具有简单易懂的逻辑和代码结构,适合初学者进行练手和学习。

\section{第二周:HEAD、GET、POST 方法的实现}
HEAD、GET和POST方法是HTTP协议中常用的请求方法,它们分别用于获取资源、查询资源和提交数据。在Web Server的实现中,可以通过定义路由和处理函数来实现这些方法。其中,HEAD方法与GET方法类似,但只返回资源的头部信息,不返回具体内容;GET方法用于获取资源并返回具体内容;而POST方法则用于向Web Server提交数据,并根据数据生成新的资源或进行其他操作。通过实现这些方法,可以使Web应用程序具有更加完善的功能和更加灵活的业务逻辑。

\section{第三周:HTTP 的并发请求的实现}
HTTP的并发请求实现可以通过多线程、多进程或协程等方式来实现。在Web Server中,可以通过线程池或异步框架等方式来处理并发请求,以提高系统的吞吐量和响应速度。在进行并发请求实现时,需要注意线程安全和资源竞争等问题,同时也需要考虑负载均衡和系统可扩展性等方面的问题,以确保系统的性能和可用性。

\section{第四周:多个客户端的并发处理}
多个客户端的并发处理可以使用多路复用技术,如select、poll、epoll等,实现单线程同时处理多个客户端请求。也可以通过多线程或多进程等方式实现并发处理,以提高系统的吞吐量和响应速度。在多客户端并发处理时,需要注意线程安全和资源竞争等问题,并且采取适当的负载均衡策略,以确保整个系统的性能和可用性。